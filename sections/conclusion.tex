\section{Conclusion}
We conduct an evaluation of design choices in chest x-ray classification and find that suprisingly almost no design choices significantly improve performance over standard image classification baselines. Moreover even when considering the best possible versions of domain-specific design choices, we find that domain-independent models are often statistically indistinguishable from these domain-specific models. The contrast between these results and the performance reported in the literature emphasizes the importance of comparison of new methods to strong baselines and robust evaluation of uncertainty in performance. Moreover our results suggest that additional work is required to understand how we can best leverage the domain-specific properties of chest x-ray classification datasets to improve model performance. We hope that our work will serve to encourage future research in this direction.
% In this work we have evaluated domain-specific models in chest x-ray classification across a multitude of methods and several datasets. We find that after accounting for uncertainty in test AUROC estimates domain independent models often produce performance indistinguishable from domain-specific models. These results hold even when we consider diverse ensembles and when we control for all possible confounding factors in a method. 

% Our results suggest that there is still more work to be done if we want to leverage the properties unique to the chest x-ray domain to improve model performance. When conducting future research in this direction we should take care to compare to strong domain-independent baselines and evaluate across several datasets to ensure new methods are truly improving upon what we can do with generic baselines. 

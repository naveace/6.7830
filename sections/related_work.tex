\section{Related Work}
Development of domain-specific methods for chest x-ray classification has been enabled by the creation of chest x-ray benchmarks. The Chest X-ray 14 benchmark \citep{wang2017chestx} created by the NIH was one of the first chest x-ray benchmarks that enabled the development of chest x-ray specific deep learning methods. Since its creation additional benchmarks, most notably the CheXpert and MIMIC benchmarks \citep{irvin2019chexpert,johnson2019mimic}, have been released, enabling comparison of model performance across different benchmarks. The CheXpert benchmark was released as part of a competition which prompted the development of some of the domain-specific methods we analyze in this paper.

Among the best performing methods in the CheXpert competition were \citet{yuan2020large}, \citet{pham2021interpreting}, and \citet{ye2020weakly}. \citet{yuan2020large} proposed a new loss function that allows for direct maximization of AUROC and achieved first place on the CheXpert competition with it. \citet{pham2021interpreting} achieved second place by proposing a method that uses a medical hierarchy among labels to faciliate training in a manner that allows for prediction of child classes conditional on parent classes being true \citep{chen2019deep}. The third strongest method was that of \citet{ye2020weakly} who proposed PCAM, a novel method for pooling that utilizes a probabilistic version of class activation maps to achieve higher performance and more interpretable results. 

Other methods specific to the chest x-ray domain exist including \citet{kamal2022anatomy}, \citet{yan2018weakly}, \citet{chen2019lesion}, and \citet{guan2018diagnose}. While we do not investigate these methods in this paper, we think applying our analysis to them would be a fruitful direction for future work.
% Among the best performing methods in the CheXpert competition were \citet{yuan2020large}, \citet{pham2021interpreting}, and \citet{ye2020weakly}. \citet{rajpurkar2017chexnet}'s CheXNet was able to achieve performance on Chest X-ray 14 that exceeded that of a human radiologist for some classes. Other methods specific to the chest x-ray domain exist including \citet{kamal2022anatomy}, \citet{yan2018weakly}, \citet{chen2019lesion}, and \citet{guan2018diagnose}. While we do not investigate these methods in this paper, we think applying this same analysis to them would be a fruitful direction for future work.

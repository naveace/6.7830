\documentclass{article}

% if you need to pass options to natbib, use, e.g.:
%     \PassOptionsToPackage{numbers, compress}{natbib}
% before loading neurips_2020

% ready for submission
% \usepackage{neurips_2020}

% to compile a preprint version, e.g., for submission to arXiv, add add the
% [preprint] option:
%     \usepackage[preprint]{neurips_2020}

% to compile a camera-ready version, add the [final] option, e.g.:
%     \usepackage[final]{neurips_2020}

% to avoid loading the natbib package, add option nonatbib:
\usepackage[preprint]{neurips_2020}

\usepackage[utf8]{inputenc} % allow utf-8 input
\usepackage[T1]{fontenc}    % use 8-bit T1 fonts
\usepackage{hyperref}       % hyperlinks
\usepackage{url}            % simple URL typesetting
\usepackage{booktabs}       % professional-quality tables
\usepackage{amsfonts}       % blackboard math symbols
\usepackage{amsmath}       % blackboard math symbols
\usepackage{nicefrac}       % compact symbols for 1/2, etc.
\usepackage{microtype}      % microtypography
\usepackage{enumitem}
\usepackage{graphicx}
\usepackage{subcaption}
\usepackage{adjustbox}
\usepackage{marginnote}
\usepackage{pifont}
\usepackage{booktabs}
\usepackage{array}
\usepackage{graphicx}
\usepackage{makecell}
\usepackage{multirow}
\usepackage[usenames,dvipsnames]{xcolor}
\usepackage{tikz}

% \newcommand*\colorcheck[1]{%
%   \expandafter\newcommand\csname #1check\endcsname{\textcolor{#1}{\ding{52}}}%
% }


\newcommand{\greencheck}{{\color{Green}\ding{52}}}
\newcommand{\redx}{{\color{Red}\ding{55}}}


\usepackage{graphicx}
\usepackage{booktabs}
\usepackage{caption}
\usepackage{subcaption}

\usetikzlibrary{calc}
\usepackage{bbm}

% \usepackage{multirow} 
\usepackage{booktabs}
\usepackage{longtable} 
\usepackage{makecell, multirow}

\usepackage{adjustbox}

\usepackage{array}
\usepackage{bbm}

\newcommand{\nih}{\textsc{NIH}}
\usepackage[colorinlistoftodos]{todonotes}

\newcommand{\chexpert}{\textsc{CheXpert}}
\newcommand{\mimic}{\textsc{MIMIC}}
\newcommand{\spur}{\textsc{Spurious}}
\newcommand{\full}{\textsc{Standard}}
\newcommand{\fg}{\textsc{Foreground}}
\newcommand{\lf}[1]{``#1''}
\usepackage{dsfont}

\usepackage{amsmath}
% \newcommand{\logan}[1]{\marginnote{\textcolor{blue}{\textbf{Logan:} #1}}}
% \newcommand{\logan}[1]{%
%   \marginnote{\color{orange}\framebox{\textcolor{black}{Logan}}}[0pt]\textcolor{black}{#1}%
% }


\newcommand{\logan}[1]{%
  \todo{\textbf{Logan}: #1}
}
\newcommand{\evan}[1]{%
  \todo{\textbf{Evan}: #1}
}

\newcommand{\robf}{robust}
\newcommand{\spurf}{spurious}
\newcommand{\robfc}{Robust}
\newcommand{\spurfc}{Spurious}


\newcommand{\bayesboth}{\hat{Y}}
\newcommand{\bayess}{\hat{Y}_\mathrm{Spur}}
\newcommand{\bayesr}{\hat{Y}_\mathrm{Rob}}


% \newcommand{\indep}{\perp \!\!\! \perp}

% \newcommand\indep{\protect\mathpalette{\protect\independenT}{\perp}}
% \def\independenT#1#2{\mathrel{\rlap{$#1#2$}\mkern2mu{#1#2}}}

\newcommand{\indep}{\perp\!\!\!\perp}
\newcommand{\nindep}{\not\!\indep{}}

\newcommand{\spurlogs}{\mathrm{logits}_{\mathrm{\spurf{}}}}
\newcommand{\stanlogs}{\mathrm{logits}_{\mathrm{standard}}}

\newcommand{\Ddeploy}{D_{d_\mathrm{deploy}}}
\newcommand{\Dint}{D_{d_\mathrm{internal}}}
\newcommand{\Dext}{D_{d_\mathrm{external}}}
\newcommand{\bern}{\mathrm{Bernoulli}}
\newcommand{\normal}{\mathcal{N}}
\newcommand{\lambdaorig}{{\lambda_{\mathrm{train}}}}

\DeclareMathOperator*{\argmax}{arg\,max}
\DeclareMathOperator*{\argmin}{arg\,min}
\DeclareMathOperator{\E}{\mathbb{E}}
\DeclareMathOperator{\Prob}{\mathrm{P}}
% \DeclareMathOperator{\ind}{\mathbb{1}}
\newcommand{\ind}[1]{\mathds{1}{\left\{#1\right\}}}

\DeclareMathOperator*{\xs}{x_{spur}}
\DeclareMathOperator*{\xr}{x_{rob}}
\usepackage{nicefrac}

\usepackage{xstring}
\usepackage{xspace}
\usepackage{ifthen}


\newcolumntype{R}[2]{%
    >{\adjustbox{angle=#1,lap=\width-(#2)}\bgroup}%
    l%
    <{\egroup}%
}
\newcommand*\rot{\multicolumn{1}{R{40}{1em}}}% no optional argument here, please!
\newcommand*\rotff{\multicolumn{1}{R{55}{1em}}}% no optional argument here, please!
\newcommand*\rothalf{\multicolumn{1}{R{90}{1em}}}% no optional argument here, please!

\newcolumntype{W}[1]{>{\raggedleft\arraybackslash}p{#1}}

\bibliographystyle{unsrtnat}

\title{A Tutorial on Probabilistic PCA}

% The \author macro works with any number of authors. There are two commands
% used to separate the names and addresses of multiple authors: \And and \AND.
%
% Using \And between authors leaves it to LaTeX to determine where to break the
% lines. Using \AND forces a line break at that point. So, if LaTeX puts 3 of 4
% authors names on the first line, and the last on the second line, try using
% \AND instead of \And before the third author name.

\author{%
  Evan Vogelbaum\\
  MIT\\
  \texttt{evanv@mit.edu} \\
}

\begin{document}

\maketitle

\begin{abstract}
  % I want to say something along the lines of "there are lots of methods out there, 
% but they are all designed for some specific task and we don't know whether they are the "best" across different architectures and datasets
% Medical machine learning has seen a surge of popularity in recent years, with chest X-ray classification being one of the most prominent examples.
% While many new methods have been proposed as advancing the state of the art for chest X-ray classification,
% it is unclear whether these methods represent fundamental advancements in the field or are simply able to perform marginally better on a specific dataset.
% In this work we present a systematic and standardized study of the performance of 120 chest X-ray classification methods across several common benchmarks.
% We find that when accounting for uncertainties at both model and data levels, all methods are comparable to that of a simple baseline classifier.
% Our findings underscore the importance of controlled comparison and examining the consistency of method performance in chest X-ray classification to ensure the development of truly generalizable methods.
While recent models for chest x-ray classification have outperformed human radiologists on chest x-ray benchmarks, it is unclear what design choices have led to this performance. To better understand the effect of these choices, we conduct a comparison of design choices across a several components relevant to chest x-ray classification models. We find that the impact of common design choices on performance is minimal. Furthermore domain-specific design choices provide no significant boost over domain-independent baselines. Our findings underscore the need for further exploration of what design choices are most effective in the chest x-ray domain.

\end{abstract}

\input

\section{Introduction}
% \evan{Citations will be added}
Models in medical ML match human radiologists on benchmarks of chest x-ray classification \citep{pham2021interpreting, yuan2020large, ye2020weakly, kamal2022anatomy,irvin2019chexpert,johnson2019mimic}. However the specific model design choices that have been key to their performance are greatly varied... some use medical hierarchies over labels \citep{pham2021interpreting}, others focus on optimization of medical performance metrics\citep{yuan2020large}, and still others have utilized specialized architectures designed to be aware of human anatomy\citep{kamal2022anatomy}. Furthermore the heterogeneity of other design choices such as training procedure or backbone architecture makes it challenging to directly compare the improvement these methods offer.

\textbf{Contributions:} We evaluate a number of design choices and find that there are almost \emph{no} choices that consistently improve model performance over baseline image classification models. Specificially, we train models on three standard chest x-ray classification benchmarks over a grid of these choices and find that:
\begin{itemize}
    \item \textbf{Domain-\emph{independent} methods match domain-\emph{specific} ones:} We find models using only domain-independent design choices are statistically indistinguishable from the best domain-specific models on two of our three datasets and highly competitive on the third.
    \item \textbf{Individual design choices (almost) \emph{never} improve performance:} \emph{No} single design choice significantly improves the performance of a standard ImageNet model across all three datasets. Across all models, we find that very few design choices significantly \emph{increase or decrease} performance across all three datasets, and that those which do produce changes in performance of less than 0.5\% AUROC.
    % \item \textbf{New Methods Don't Change Clinical Decisions } TODO
\end{itemize}

These results emphasize the need to compare to strong baselines and robustly estimate the uncertainty in improvements over these baselines when proposing new design choices. They further suggest that additional work is still required to understand what design choices are key to improving model performance in the chest x-ray domain. 
% While generic image benchmarks often serve as a good starting point for model selection, it is common sense that domain-specific benchmarks should allow us to find better models for a given domain. Although the creation of chest x-ray datasets has prompted the development of models designed specifically for the chest x-ray domain, there has yet to be a systemic evaluation of what modeling choices work well in this domain and to what extent domain-specific models improve performance over generic models.

% In this work we seek to conduct this evaluation through a case study of domain-specific and domain-independent modeling choices in chest x-ray classification. We analyze perfomance across 150 different modeling pipelines and three common chest x-ray classification datasets. Our main contributions are as follows:
% \begin{itemize}
%     \item We find that no single modeling choice significantly improves the performance of a standard ImageNet model across all three datasets.
%     \item Across all modeling pipelines, we find that very few modeling choices significantly \emph{increase or decrease} performance across all three datasets, and that those which do often produces changes in performance of less than 0.5\% AUROC.
%     \item We find that a fully domain-independent model is statistically indistinguishable from the best domain-specific models on 2 of our three datasets and highly competitive on the third.
% \end{itemize}

% These findings suggest that additional work is still required to understand how we can best leverage the unique properties of the chest x-ray domain to improve model performance. It further implies that we may want to further study whether we can use standard chest x-ray datasets to distinguish between performance of different modeling choices.

\section{Methods}
\label{sec:methods}
\subsection{Construction of Methods}
Our goal is to understand which design choices consistently improve performance in chest x-ray classification and whether domain-specific choices outperform domain-independent ones. This requires that we identify a wide variety of both domain-specific and domain-independent design choices which lead to competitive results. We also want to understand the impact of a design choices independent of the other choices we make for the components of a model. To that end we construct our \emph{methods} combinatorially, choosing one of the following options for each component of the method. We use star ($\star$) to indicate chest x-ray domain-specific choices.
\begin{enumerate}
    \item \textbf{Loss Function}:
        \begin{itemize}
            \item Standard Binary Cross Entropy loss (BCE) 
            \item Focal Loss (Focal): \citet{lin2017focal}'s loss function designed to handle class imbalance.
            \item[$\star$] CheXNet BCE (CheXNet): BCE loss with weights designed to balance the binary labels for each class as proposed by \citet{rajpurkar2017chexnet}.
            % Because of widely varying label imbalances across classes we also consider focal loss \citep{lin2017focal} as a loss function. This loss function has seen wide success in object detection when there is extreme class imbalance and serves as a second strong baseline to compare chest x-ray specific methods against. \evan{note that LibAUC compared to this as additional motivation?}
            \item[$\star$]Deep AUC Maximization (DAM): \citet{yuan2020large}'s novel loss function designed to directly maximize AUROC, the standard metric for chest x-ray classification.
            % . Following their work we first pretrain each model using BCE loss and then fine tune each class seperately using the DAM loss function. The final model makes predictions for each class seperately using the fine tuned models. \evan{Should these details go in appendix?}
            \item[$\star$]  Hierarchical Training (Hierarchical): \citet{pham2021interpreting}'s hierarchical training method which uses a medical hierarchy over labels.
            %. Following their work we pretrain conditionally using masked labels and then fine tune using the novel loss function of \citet{chen2019deep}. We make predictions as described in \citet{pham2021interpreting} using Bayes rule to get unconditional probabilities.
        \end{itemize}
    \item \textbf{Backbone}: We consider 5 different backbones for our methods: ResNet-18, ResNet-50, DenseNet-121, VGG16, and VGG19 with Batch Norm. All of these backbones have been used in chest x-ray classification \citep{moses2021deep}.
    \item \textbf{Pooling Function}: 
        \begin{itemize}
            \item[$\star$] PCAM: We use the PCAM module \citep{ye2020weakly}, which was designed for chest x-ray classification, as the pooling function.
            \item Standard: We use the standard pooling function for the backbone of the method as implemented in \texttt{torvision} \citep{paszke2019pytorch}.
        \end{itemize}
    \item \textbf{Data Augmentation}: We consider three data augmentation schemes: 
        \begin{itemize}
            \item No Augmentation (No Aug): No data augmentation is used.
            \item Random Translate \& Cutout (CIFAR): We randomly translate and cutout a portion of the image. This augmentation is representative of a standard augmentation scheme for the CIFAR-10 dataset \citep{krizhevsky2009learning}. 
            \item Color Jitter with Random Crop \& Mixup (ImageNet): We use color jitter, random crop and mixup \citep{zhang2017mixup} as a representative of a standard augmentation scheme for the ImageNet dataset \citep{deng2009imagenet}.
        \end{itemize} 
    % No Augmentation (No Aug), Random Translate \& Cutout (RT+CO), and Color Jitter with Random Crop \& Mixup (CJ+RC+MU)
   \end{enumerate}

\subsection{Datasets}

We evaluate each method on the CheXpert \citep{irvin2019chexpert}, MIMIC-CXR \citep{johnson2019mimic}, and Chest X-ray 14 \citep{wang2017chestx} datasets\footnotemark\footnotetext{We follow \citet{pham2021interpreting} for a hierarchy for the CheXpert and MIMIC-CXR datasets and construct our own medically valid hierarchy for Chest X-ray 14 dataset. All three are included in the appendix.}. The CheXpert dataset uses different labeling processes for its training and validation splits. To avoid distributional mismatch between the data we train on and the data we use for hyperparameter tuning we randomly re-split the CheXpert train set, keeping 80\% of the original for training and using 10\% as a validation set and 10\% as a test set. We ensure that splitting is done so as to avoid overlap between patients across the partitions. For the MIMIC-CXR and Chest X-Ray 14 datasets, we use the given validation and test sets as they do not have this distributional mismatch issue. For both CheXpert and MIMIC-CXR we ignore uncertainty labels (the U-IGNORE strategy in \citet{irvin2019chexpert}) for ease of implementation and comparison with the Chest X-ray 14 dataset which does not have uncertainty labels.
\subsection{Training and Evaluation}
For each method, we perform a grid search over the learning rate and weight decay using a 6x6 grid. We also grid search over  other parameters relevant to the method such as $\gamma$ for methods that use Focal Loss as the loss function. For methods that require finetuning, such as those using the Hierarchical or DAM loss functions, we first conduct a grid search over models for the pretraining step and select the best pretrained model to perform a second grid search on for finetuning. To reduce computational load we prune all runs that cannot achieve an average of 0.6 AUROC on the validation set after 2 epochs. These intensive grid searches ensure that for every method we consider we get close to the best possible performance that the method can achieve on each dataset.

On all datasets we evaluate performance using AUROC as is standard in chest x-ray classification. On CheXpert and MIMIC-CXR, we use the average performance of the 5 classes used in the test set for the CheXpert competition as our score for a method. On Chest X-ray 14, we use the average AUROC across all 14 labels. To evaluate uncertainty in our scores we recompute the test AUROC of each method using 1000 bootstrap samples over the test partition for each dataset. 
%We also re-run our best models with 20 different random seeds to obtain model level uncertainty. \evan{This is still TODO}



\section{Results}
\subsection{Motivating Example: Improving a Standard ImageNet Baseline}
\label{subsec:imagenet-baseline}
\begin{figure}
    \centering
    \includegraphics[width=\textwidth]{figures/14.00-ehv-motivating-plot.png}
    \caption{Trying to improve a standard ImageNet baseline. Each panel shows the change in performance achieved by swapping out one component of a standard ImageNet baseline (No Augmentation, ResNet-50, BCE Loss, Standard Pooling) with a different design choice. Remarkably there is \emph{no single change we can make} that results in a statisticaly significant improvement in AUROC across all three benchmarks. }
    \label{fig:imagenet-baseline}
\end{figure}
We begin our evaluation of design choices in chest x-ray classification by considering a simple task: improving a standard ImageNet model. A ResNet-50 trained via BCE loss without special data augmentation or pooling is a standard baseline in general image classification \citep{developers2016pytorch}, however given the development of domain-specific methods and practices for chest x-ray classification we might expect that we can improve on this baseline substantially through changes in each of its components. Figure \ref{fig:imagenet-baseline} shows how performance changes when we consider changing the data augmentation, pooling method, backbone, and loss function for this simple baseline model. Each row shows the difference in performance between the baseline model and a model which uses a different choice for one of these components, keeping all the other components fixed. 

Remarkably, there is \emph{no change we can make in a single component} which significantly improves performance across all three chest x-ray datasets. Moreover it is notable that one domain-specific choice--hierarchical training--actually consistently performs \emph{worse} than this simple baseline. This example suggests that there may be a need for more careful evaluation of the performance benefits given by design choices used in chest x-ray classification.

\subsection{Comparing Design Choices}
\label{subsec:controlled-comparison}
\begin{figure}
    \centering
    \includegraphics[width=\textwidth]{figures/12.08-ehv-controlled-comparison.png}
    \caption{Average Treatment Effect of design choices over baseline choices. We conduct a controlled comparison of design choices by evaluating the average change in AUROC across all methods when we swap out a baseline choice. Almost no design choices consistently increase or decrease AUROC across all three datasets. Those that do (PCAM pooling, CheXNet loss type, ResNet-18 and VGG-16 backbone, and CIFAR data augmentation) produce changes that are often less than 0.5\% AUROC meaning they have a very small treatment effect. }
    \label{fig:controlled-comparison}
\end{figure}
Motivated by our results in Section \ref{subsec:imagenet-baseline}, we aim to conduct a more systemic evaluation of how common design choices improve over standard image classification baselines. Because we train a model for every choice of (data augmentation, pooling, backbone, loss type) we are able to evaluate the difference in AUROC between a model that uses a baseline choice for one of these components and a model that uses a different choice for that component \emph{leaving all other choices fixed}. This gives us a \emph{treatment effect} of that design choice over the baseline choice. We can then compute an \emph{average treatment effect} by averaging this treatment effect over all choices for the other components. We do exactly this in Figure \ref{fig:controlled-comparison}. Our baseline choices are the standard setup described in the last section: No Augmentation, Standard Pooling, ResNet-50, and BCE Loss. In each row we see the average treatment effect of different choices bootstrapped over 1000 samples of the test dataset for each benchmark.

The most striking result from this figure is that almost no design choices have a statistically significant effect on AUROC across all three benchmarks. Among loss functions, only CheXNet loss has a consistently statistically significant effect on AUROC, and even then the effect is small (< 0.5\% AUROC). Furthermore it is worth noting CheXNet loss is the least domain-specific of the domain-specific loss functions we consider, effectively just a weighted version of BCE loss. A similiar story holds for the other choices that have a statistically significant effect on AUROC across all three benchmarks. The only exception is VGG-16, but this is because it seems to be a particularly poor choice of architecture for chest x-ray classification, not because it leads to significant improvements. Especially notable is the fact that the DenseNet-121 architecture, which has been noted to be a strong choice for chest x-ray classification in the dicta of many past works \citep{irvin2019chexpert,pham2021interpreting,rajpurkar2017chexnet}, provides essentially no treatment effect. 

That design choices designed for the chest x-ray domain such as Hierarchical Loss, DAM, and PCAM produce such small treatment effects is especially surprising. When engineered to produce peak performance, these methods have matched human radiologists on some tasks \citep{irvin2019chexpert}. However, our results suggest that these methods in isolation are not significantly better than a standard image classification baseline.

\subsection{Comparing domain-specific choices to baselines ones}
\begin{figure}
    \centering
    \includegraphics[width=\textwidth]{figures/06.04-ehv-best-method-comparison-remake.png}
    \caption{
        Comparing the best versions of domain-specific design choices with domain-independent baselines. For each domain-specific design choice we compare the best method we can construct using that choice with the best method we can construct using only domain-independent choices. We find that on CheXpert and MIMIC the domain-independent baseline is just as strong as the best methods we can construct using domain-specific choices. On Chest X-ray 14 the domain-independent baseline is not quite as strong as the best methods using domain-specific choices, but it is still comparable to Hierarchical Loss and it is within 1\% AUROC of other methods.
    }
    \label{fig:best-method-per-choice}
\end{figure}
Results from Section \ref{subsec:controlled-comparison} suggest that domain-specific design choices may not have a significant effect on AUROC. However in that section we computed the performance of domain-specific design choices across all configurations of design choices. In order to fairly evaluate these design choices it is also important to compare the \emph{best} performance we can achieve using domain-specific design choices with the performance of domain-independent baselines. We do exactly this in Figure \ref{fig:best-method-per-choice}.

For each domain-specific design choice we compare the best method we can construct using that choice with the best method we can construct using only domain-independent choices (Baseline). In order to evaluate the source of improvement of a domain-specific choice over a baseline, we also include the best models we can construct using domain-independent choices for the same component but allowing domain-specific choices for the other components (e.g. the best model we can make that uses Focal Loss while also allowing the use of PCAM). 

On CheXpert and MIMIC, we find that suprisingly even the best methods we can construct using domain-specific design choices are not significantly better than the best methods we can construct using only domain-independent choices. Furthermore, although on Chest X-ray 14 the baseline is not quite as strong as the domain-specific methods, when we allow for domain-specific choices in other components we see that domain-independent choices are comparable in performance. These results suggest that we have yet to determine how best to leverage domain-specific knowledge in chest x-ray classification to improve performance over baselines.

% \subsection{Comparing Attribute Choices}
% The most straightforward way to compare the performance of domain specific attribute choices to domain independent choices is to consider the best methods we can construct holding each of these choices fixed. Because we have chest x-ray specific choices for both pooling function and loss function, it is also useful to construct a set of \emph{baseline} methods which use only domain independent choices for these attributes. We show the best choices for these baseline methods alongside the best methods we can construct holding each attribute choice fixed for each dataset in Figure \ref{fig:best-method-per-attribute}.

% On CheXpert and MIMIC one striking conclusion we can draw is that the baseline method is just as strong as the best methods we can construct even when using domain-specific attributes. This is surprising and suggests that existing chest x-ray specific methods may not be meaningfully better than domain independent baselines on these datasets.

% On Chest X-ray 14 we see a similiar story. Although the pure baseline method is not quite as strong when compared to the best methods using domain specific attributes, it is still comparable to Hierarchical Loss and it is within 1\% AUROC of other methods. Furthermore our setup allows us to evaluate what the cause of increased performance is for these methods.

% The only difference between the best baseline method and the best method using Focal Loss is the use of PCAM.  Comparing the best method for Focal Loss with the best methods for domain-specific loss functions, we see that once the use of PCAM is controlled for, domain-independent loss functions are just as strong as domain-specific loss functions. 

% We get the same result when considering the best methods for PCAM or Standard pooling. The only difference between the best method for Standard pooling and the best method for PCAM is the use of DAM Loss. Once this is controlled for, we see that PCAM and Standard pooling are indistinguishable in terms of performance.

% These results suggest that there is still room to improve chest x-ray classification performance by developing new domain specific methods. However, they also suggest that the current state of the art methods which are domain specific may not be as strong as they appear when compared to domain independent baselines.

% Lastly it is worth noting that Figure \ref{fig:best-method-per-attribute} also lets us analyze choices for other attributes such as Backbone and Data Augmentation. In contrast to prior work \citep{irvin2019chexpert,pham2021interpreting} we find little difference between backbones on CheXpert and MIMIC and suprisingly find VGG-19 to be one of the best backbones on Chest X-ray 14. We also find that data augmentation only seems to matter on Chest X-ray 14. While these results are not the primary focus of our study, the contrast with prior work suggests that the best choices for these attributes may be dependent on the other choices made in the method being considered and that additional work may be needed to understand the best choices for these attributes.

% \begin{figure}
%     % Make three figures in the following arrangment
%     %  A  B C
%     \centering
%     \includegraphics[scale=0.6]{figures/06.03-ehv-bootstrap-mega-figure.png}
%     \caption{\textbf{Comparison of Best Method Per Attribute}. For each dataset we take the best method for each choice of the attribute considered and show the test AUROC with confidence intervals. The baseline is restricted to only domain independent attribute choices. On CheXpert and MIMIC we see that the baseline is just as strong as the best methods even for domain-specific attributes like Hierarchical, DAM, and PCAM. On Chest X-ray 14 while the baseline alone is not quite as strong, when using PCAM domain independent loss functions are just as strong as domain-specific loss functions, and when using domain specific loss functions standard pooling is just as strong as PCAM.} 
%     \label{fig:best-method-per-attribute}
% \end{figure}

% \begin{figure}
%         \centering
%     \includegraphics[scale=0.6]{figures/09.02-ensemble-comparison.png}
%     \caption{\textbf{Comparison of Best Methods Ensembled}. For each dataset we consider an ensemble of the best methods we can construct holding each of our Loss and Pooling attribute choices fixed as well as an ensemble of our best baseline methods. We also show the results from Figure \ref{fig:best-method-per-attribute} using transparent colors for comparison. We see similiar conclusions as in Figure \ref{fig:best-method-per-attribute}: on CheXpert and MIMIC the baseline ensemble is just as strong as domain-specific ensembles. It is also worth noting that ensembling provides a consistent boost in performance for all methods, in line with previous work.} 
%     \label{fig:best-method-per-ensemble}

% \end{figure}
% \subsection{Ensembling}
% Ensembling of diverse methods is a common approach in chest x-ray classification and many of the best results have been reported based on ensembles of multiple methods holding on specific attribute (such as a loss function) fixed \citep{pham2021interpreting,yuan2020large,ye2020weakly}. Thus it is important to see whether the conclusion we draw from the best individual methods change when we consider an ensemble of methods.

% In Figure \ref{fig:best-method-per-ensemble} we show the results of ensembling the best methods we can construct holding each attribute choice fixed as well as ensembling the best baseline methods. We see that the conclusions we draw from Figure \ref{fig:best-method-per-attribute} hold even when ensembling is used. On CheXpert and MIMIC the baseline ensemble is just as strong as domain-specific ensembles. 

% On Chest X-ray 14 one important conclusion we can draw is that an ensemble of baseline methods is just as strong as the best individual methods which use domain-specific attributes. This shows that we can match the performance of the best domain-specific methods by simply ensembling the best domain-independent methods.

% We also see that for Chest X-ray14 ensembles for domain independent loss functions are just as strong as the ensembles for domain specific loss functions when we control for whether PCAM use is allowed. Similiarly we see that the ensembles for PCAM and Standard pooling are indistinguishable when we control for whether DAM use is allowed.


% Finally, comparing the ensembles with the best individual version of each method, we see that in line with previous work, ensembling provides a consistent performance boosts for all methods. This suggests that while existing domain-specific methods for chest x-ray classification may not always provide a significant boost over domain-independent baselines, ensembling remains a promising way to boost performance. 

% \begin{figure}[]
%     % Make three figures in the following arrangment
%     %  A  B C
%         \includegraphics[width=\textwidth]{figures/12.04-ehv-loss-type-differences.png}
%     \caption{\textbf{Controlled Differences Between Loss Functions}. For each choice of (backbone, pooling type, data augmentation) we consider the difference between using a domain specific loss function (Hierarchical, DAM, CheXNet) and a domain independent loss function (Focal). We then compute the distribution of differences across all both choices of (backbone, pooling type, data augmentation) and bootstrapped samples of the test dataset. For each dataset we show the median difference with 95\% confidence intervals. In all cases we cannot conclude that the domain specific loss functions are better than the domain independent loss functions.}
%     \label{fig:controlled-loss}
% \end{figure}

% \begin{figure}[]
%     % Make three figures in the following arrangment
%     %  A  B C
%         \includegraphics[width=\textwidth]{figures/12.04-ehv-pcam-differences.png}
%     \caption{\textbf{Controlled Differences Between Pooling Types}. For each choice of (loss function, backbone, data augmentation) we consider the difference between using PCAM and Standard pooling. We then compute the distribution of differences across all both choices of (loss function, backbone, data augmentation) and bootstrapped samples of the test dataset. For each dataset we show the median difference with 95\% confidence intervals. In all cases we cannot conclude that PCAM is significantly better than Standard pooling.}

%     \label{fig:controlled-pcam}
% \end{figure}

% \subsection{Controlled Evaluation}
% The previous two analyses have shown that when we consider the best methods we can construct holding each attribute choice fixed, domain independent choices are just as strong as domain specific choices. However this is not all we can explore with our setup. Because we create methods through a grid of attribute choices, for every method we can examine what would happen if we changed just one of the attribute choices from domain independent to domain dependent. This effectively allows us to compute the treatment effect of each attribute choice on performance.

% As in our previous two analyses, we must remember that measurements of performance difference on the test set are only estimates of the difference in performance on the population. We can again use the bootstrap to estimate the uncertainty in our estimates. For each method, we first sample a new test set and then estimate the difference in performance between the domain independent and domain specific versions of the attribute choice. We repeat this process 1000 times for each method to get a distribution of performance differences. We then compute the median difference and 95\% confidence intervals for each dataset.

% We use this procedure to estimate the difference in performance between domain independent and domain specific loss functions in Figure \ref{fig:controlled-loss}. We use Focal Loss as our domain independent loss choice and compare it to the domain dependent loss functions (Hierarchical, DAM, CheXNet) for each dataset. On all three datasets and for all three loss functions, we see that their improvements over focal loss are not statistically different from zero. This shows that even in a controlled setting, we cannot conclude that domain specific loss functions are better than domain independent loss functions.

% We use the same procedure to estimate the difference in performance between PCAM and Standard pooling in Figure \ref{fig:controlled-pcam}. Again on all three datasets, the difference in performance between PCAM and Standard pooling is not statistically different from zero.

% These results are consistent with our previous analyses and suggest that their is still room to improve chest x-ray classification performance through domain specific methods.


\section{Related Work}
Development of domain-specific methods for chest x-ray classification has been enabled by the creation of chest x-ray benchmarks. The Chest X-ray 14 benchmark \citep{wang2017chestx} created by the NIH was one of the first chest x-ray benchmarks that enabled the development of chest x-ray specific deep learning methods. Since its creation additional benchmarks, most notably the CheXpert and MIMIC benchmarks \citep{irvin2019chexpert,johnson2019mimic}, have been released, enabling comparison of model performance across different benchmarks. The CheXpert benchmark was released as part of a competition which prompted the development of some of the domain-specific methods we analyze in this paper.

Among the best performing methods in the CheXpert competition were \citet{yuan2020large}, \citet{pham2021interpreting}, and \citet{ye2020weakly}. \citet{yuan2020large} proposed a new loss function that allows for direct maximization of AUROC and achieved first place on the CheXpert competition with it. \citet{pham2021interpreting} achieved second place by proposing a method that uses a medical hierarchy among labels to faciliate training in a manner that allows for prediction of child classes conditional on parent classes being true \citep{chen2019deep}. The third strongest method was that of \citet{ye2020weakly} who proposed PCAM, a novel method for pooling that utilizes a probabilistic version of class activation maps to achieve higher performance and more interpretable results. 

Other methods specific to the chest x-ray domain exist including \citet{kamal2022anatomy}, \citet{yan2018weakly}, \citet{chen2019lesion}, and \citet{guan2018diagnose}. While we do not investigate these methods in this paper, we think applying our analysis to them would be a fruitful direction for future work.
% Among the best performing methods in the CheXpert competition were \citet{yuan2020large}, \citet{pham2021interpreting}, and \citet{ye2020weakly}. \citet{rajpurkar2017chexnet}'s CheXNet was able to achieve performance on Chest X-ray 14 that exceeded that of a human radiologist for some classes. Other methods specific to the chest x-ray domain exist including \citet{kamal2022anatomy}, \citet{yan2018weakly}, \citet{chen2019lesion}, and \citet{guan2018diagnose}. While we do not investigate these methods in this paper, we think applying this same analysis to them would be a fruitful direction for future work.

\section{Conclusion}
We conduct an evaluation of design choices in chest x-ray classification and find that suprisingly almost no design choices significantly improve performance over standard image classification baselines. Moreover even when considering the best possible versions of domain-specific design choices, we find that domain-independent models are often statistically indistinguishable from these domain-specific models. The contrast between these results and the performance reported in the literature emphasizes the importance of comparison of new methods to strong baselines and robust evaluation of uncertainty in performance. Moreover our results suggest that additional work is required to understand how we can best leverage the domain-specific properties of chest x-ray classification datasets to improve model performance. We hope that our work will serve to encourage future research in this direction.
% In this work we have evaluated domain-specific models in chest x-ray classification across a multitude of methods and several datasets. We find that after accounting for uncertainty in test AUROC estimates domain independent models often produce performance indistinguishable from domain-specific models. These results hold even when we consider diverse ensembles and when we control for all possible confounding factors in a method. 

% Our results suggest that there is still more work to be done if we want to leverage the properties unique to the chest x-ray domain to improve model performance. When conducting future research in this direction we should take care to compare to strong domain-independent baselines and evaluate across several datasets to ensure new methods are truly improving upon what we can do with generic baselines. 


% \section*{Broader Impact}

% \section*{References}

% \section*{Acknowledgments}
% \begin{ack}
%   Thank you to..
% \end{ack}

\bibliography{bibliography/fmt_bib}

\newpage
\appendix
\section*{Appendix}

\subsection*{Hierarchies for CheXpert, MIMIC, and Chest X-ray 14}
Our hierarchy for CheXpert and MIMIC is taken from \citet{pham2021interpreting} who in turn use the hierarchy given in \citet{irvin2019chexpert}. For Chest X-ray 14 we use our own hierarchy derived from our knowledge of the datasets. We describe each hierarchy below by giving the parent-child relationships. If $A\rightarrow B$ it means that $A$ is a parent of $B$ in the hierarchy. Classes with no parents are assumed to have their parent as the root node.
\subsubsection*{CheXpert and MIMIC}
Enlarged Cardiomediastinum $\rightarrow$ Cardiomegaly

Lung Opacity $\rightarrow$ Edema

Lung Opacity $\rightarrow$ Consolidation

Lung Opacity $\rightarrow$ Pneumonia

Consolidation $\rightarrow$ Pneumonia

Lung Opacity $\rightarrow$ Atelectasis

Lung Opacity $\rightarrow$ Lesion

\subsection*{Chest X-ray 14}
Pneumonia $\rightarrow$ Atelectasis

Pneumonia $\rightarrow$ Effusion

Nodule $\rightarrow$ Mass

Pneumonia $\rightarrow$ Consolidation

Pneumonia $\rightarrow$ Edema

Fibrosis $\rightarrow$ Pleural Thickening

\end{document}
